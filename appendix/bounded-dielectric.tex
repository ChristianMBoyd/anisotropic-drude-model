\section{The inter-electron Coulomb interaction in a bounded, anisotropic dielectric}
\label{appendix: Coulomb}

In the orthorhombic dielectric background considered in the main text, $\e=\diag(\ex,\ey,\ez)$, we can scale out the dielectric anisotropy in Gauss's law by a similar coordinate transformation to that in \note{Check reference for ``section: anisotropic chi0"} via $(x,y,z)\to(x/\sqrt{\ex},y/\sqrt{\ey},z/\sqrt{\ez})$.  As noted in \cite{Mele2001,Ivchenko2021}, however, the scale transformation leaves us with additional components of $\e$ in the potential sourced by a point charge.  For an electron point charge ($-e$) located within the bounded dielectric at $0<z'<L$, the particular solution $\phi_p$ to Gauss's law at $0<z<L$ (also within the dielectric) in a planar geometry is given by
\begin{equation}
    \label{app phip def}
    \phi_p(z) = \phi_0 \,e^{-\xi|z-z'|}
    \,\,\,,
\end{equation}
where the scaled wavevector $\xi$ was defined in \eqref{xi def} of the main text and we've collected the potential scale into
\begin{equation}
    \label{app phi0 def}
    \phi_0:= \frac{-e}{2\e_0\ez\xi}
    \,\,\,.
\end{equation}
The homogeneous solution $\phi_h$ to Gauss's law in vacuum (i.e., $z<0$ or $z>L$) can be enumerated through growing and decaying modes,
\begin{equation}
    \label{app phih vac def}
    \phi_h(z) = \phi_- e^{-|q| z} + \phi_+ e^{+|q| z}
    \,\,\,,
\end{equation}
where the $\phi_{\pm}$ are undetermined coefficients.  In the dielectric region ($0<z<L$), the homogeneous solutions remain exponentials but with the substitution $|q|\to\xi$,
\begin{equation}
    \label{app phih mat def}
    \phi_h(z) = \phi_- e^{-\xi z} + \phi_+ e^{+\xi z}
    \,\,\,.
\end{equation}

The total potential $\phi$ below the dielectric ($z<0$),
\begin{equation}
    \label{app phi< def}
    \phi(z) = \phi_< e^{+|q|z}
    \,\,\,,
\end{equation}
and above ($z>0$),
\begin{equation}
    \label{app phi> def}
    \phi(z) = \phi_> e^{-|q|z}
    \,\,\,,
\end{equation}
consists of a single, decaying homogeneous mode.  In the dielectric region ($0<z<L$), the particular solution \eqref{app phip def} and both of the growing and decaying modes in \eqref{app phih mat def} are present,
\begin{equation}
    \label{app phi pm def}
    \phi(z) = \phi_- e^{-\xi z} + \phi_+ e^{+\xi z} + \phi_0 e^{-\xi|z-z'|}
    \,\,\,.
\end{equation}
Determining the unknown homogeneous coefficients in \eqref{app phi< def}-\eqref{app phi pm def} requires use of the electrostatic boundary conditions,
\begin{align}
    \label{app BC1 def}
    \lim_{z\to z_B^-}\phi(z) &= \lim_{z\to z_B^+}\phi(z)\,\,\,,
    \\ \label{app BC2 def}
    \lim_{z\to z_B^-}\ez^-\partial_z\phi(z) &= \lim_{z\to z_B^+}\ez^+\partial_z\phi(z)
    \,\,\,,
\end{align}
where the boundary is located at $z=z_B$ and $\ez^{\pm}$ is the $z-$component of the dielectric constant above ($+$) and below ($-$) the boundary.  We have then the system of equations
\begin{align}
    \label{app z0 BC1}
    \phi_< &= 
    \phi_- + \phi_+ + e^{-\xi z'}\phi_0
    \,\,\,,
    \\
    \label{app z0 BC2}
    \phi_< &=
    \te\lp
    -\phi_- + \phi_+ + e^{-\xi z'}\phi_0
    \rp
    \,\,\,,
\end{align}
at the $z=0$ boundary and
\begin{align}
    \label{app zL BC1}
    e^{-|q|L} \phi_>  &=
    e^{-\xi L} \phi_-  + e^{+\xi L} \phi_+ + e^{-\xi L}e^{+\xi z'} \phi_0
    \,\,\,,
    \\
    \label{app zL BC2}
    -e^{-|q|L}\phi_> 
    &=
    \te\lp
    -e^{-\xi L} \phi_- + e^{+\xi L} \phi_+
    -
    e^{-\xi L} e^{+\xi z'} \phi_0 
    \rp
    \,,
\end{align}
at the $z=L$ boundary, where the effective dielectric constant $\te$ is defined in \eqref{te def} of the main text.

Solving the system of equations \eqref{app z0 BC1}-\eqref{app zL BC2} provides
\begin{align}
    \label{app phi<}
    \phi_< &=
    \phi_0
    \frac{2\te\lp
    e^{-\xi z'}+\alpha e^{+\xi z'}e^{-2\xi L}
    \rp}{\lp\te+1\rp\lp 1-\alpha^2 e^{-2\xi L}\rp}
    \,\,\,,
    \\
    \label{app phi+}
    \phi_+ &=
    \phi_0
    \frac{\alpha e^{-2\xi L}\lp
    \alpha e^{-\xi z'}+e^{+\xi z'}
    \rp}{1-\alpha^2 e^{-2\xi L}}
    \,\,\,,
    \\
    \label{app phi-}
    \phi_- &=
    \phi_0
    \frac{\alpha \lp
     e^{-\xi z'}+\alpha e^{+\xi z'}e^{-2\xi L}
    \rp}{1-\alpha^2 e^{-2\xi L}}
    \,\,\,,
    \\
    \label{app phi>}
    \phi_> &=
    \phi_0
    \frac{2\te \,e^{(|q|-\xi)L}\lp
    \alpha e^{-\xi z'} +e^{+\xi z'}
    \rp}{\lp\te+1\rp\lp 1-\alpha^2 e^{-2\xi L}\rp}
    \,\,\,,
\end{align}
where $\alpha$ is defined in \eqref{alpha def} of the main text.  The solutions for $\phi_+$ \eqref{app phi+} and $\phi_-$ \eqref{app phi-} determine the inter-electron Coulomb interaction $V(z,z')$ when both electrons are located within the dielectric ($0<z, z'<L$),
\begin{align}
    \label{app real space Coulomb}
    V(z,z') &=
    \frac{e^2}{2\e_0 \ez\xi}
    \Bigg[
    e^{-\xi |z-z'|}
    +
    \frac{\alpha e^{-\xi(z+z')}}{1-\alpha^2 e^{-2\xi L}}
    \\ \nn &+
    \frac{\alpha e^{-2\xi L}
    e^{+\xi(z+z')}}{1-\alpha^2 e^{-2\xi L}}
    +
    \frac{2\alpha^2 e^{-2\xi L}\cosh\xi(z-z')}{1-\alpha^2 e^{-2\xi L}}
    \Bigg]
    \,.
\end{align}
The result for $V(z,z')$ in \eqref{real space Coulomb} of the main text corresponds to discarding terms that are exponentially-suppressed in the macroscopic limit, $\xi L\gg 1$.  Further, the isotropic limit ($\te\to\e$ or $\ex=\ey=\ez$) reduces to the standard expression for the Coulomb interaction within a semi-infinite dielectric.\footnote{See, e.g., the usage in \cite{Jain1985_2}.}.

Writing $\chi_0(z,z')$ in terms of its cosine modes \eqref{cosine anisotropic chi0} and formally doing the same for $\chi(z,z')$ within the RPA equation \eqref{real space RPA def} leaves us with the Coulomb integrals
\begin{align}
    \label{app cosine Coulomb def}
    V(Q_n, Q_{n'}):= \int_0^L dz dz'\, V(z,z') \cos Q_n z \cos Q_{n'} z'
    \,\,\,.
\end{align}
Notably, $V(z,z')$ in \eqref{app real space Coulomb} possesses the inversion symmetry discussed in Section~\ref{section: cosine}, $z\to L-z$ and $z'\to L-z'$.  As a result, $V(Q_n,Q_{n'})$ is parity-restricted.  When $n$ and $n'$ have the same parity, the integrals in \eqref{app cosine Coulomb def} provide
\begin{align}
    \label{app cosine Coulomb}
    V(Q_n,Q_{n'}) &=
    V_0(Q_n)\Bigg[
    \frac{L}{2}\lp\delta_{n,n'}+\delta_{n,-n'}\rp
    \\ \nn &-
    \frac{\xi\lp1-\alpha\rp}{\xi^2+Q_{n'}^2}
    \frac{1\mp e^{-\xi L}}{1\mp\alpha e^{-\xi L}}
    \Bigg]
    \,\,\,,
\end{align}
where the potential scale $V_0(Q_n)$ is defined in \eqref{V0 def} of the main text and the upper (lower) sign is taken for overall even (odd) index parity.  The result stated in \eqref{cosine Coulomb} of the main text corresponds to discarding the exponentially-suppressed terms of \eqref{app cosine Coulomb} in the macroscopic limit, $\xi L\gg 1$.