\section*{Introduction}
\addcontentsline{toc}{section}{Introduction}

This paper studies an anisotropic, but homogeneous, dielectric tensor,
\begin{equation}
    \label{dielectric tensor definition}
    \e(\omega) = \bpm
    \ex(\omega) & 0 & 0
    \\ 0 & \ey(\omega) & 0
    \\ 0 & 0 & \ez(\omega)
    \epm
    \,\,\,,
\end{equation}
where each of the axial components has a Drude-like form.  In particular, the diagonal entries in $\e$ \eqref{dielectric tensor definition} each take the form
\begin{equation}
    \label{dielectric component Drude definition}
    \e_j(\omega) = \e_{j,\infty}\lp 1-\frac{\omega_{p,j}^2}{\omega\lp\omega+i\Gamma\rp}\rp\,\,\,,
\end{equation}
where $j\in(x,y,z)$, the $\e_{j,\infty}$ describe inter-band or background dielectric contributions, $\omega_{p,j}$ is the frequency of plasma oscillation along the $j$th axis,\footnote{In particular, the $\omega_{p,j}$ are the physical plasma frequencies that the plasmon would be observed at along the $j$th axis, not the unscreened value $\omega_p=\sqrt{ne^2/m\e_0}$.} and $\Gamma$ is the phenomenological scattering rate.  The choice of an isotropic scattering rate is for convenience; most of the subsequent study will focus on well-defined bulk excitations ($\omega\gg\Gamma$), where anisotropic scattering is presumed to not be the dominant effect.