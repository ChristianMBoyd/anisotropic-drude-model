\section{The anisotropic Drude model}

This paper examines an electrostatic system governed by an anisotropic and homogeneous dielectric tensor,
\begin{equation}
    \label{dielectric tensor definition}
    \e(\omega) = \bpm
    \ex(\omega) & 0 & 0
    \\ 0 & \ey(\omega) & 0
    \\ 0 & 0 & \ez(\omega)
    \epm
    \,\,\,,
\end{equation}
where the diagonal entries of $\e$ \eqref{dielectric tensor definition} resemble the Drude model,
\begin{equation}
    \label{dielectric component Drude definition}
    \e_j(\omega) = \e_{j,\infty}-\frac{\omega_{j,0}^2}{\omega\lp\omega+i\Gamma\rp}\,\,\,.
\end{equation}
In \eqref{dielectric component Drude definition}, $j\in(x,y,z)$, $\e_{j,\infty}$ phenomenologically models inter-band or background dielectric contributions oriented along the $j$th axis, $\omega_{j,0}$ is the {\it unscreened} frequency of plasma oscillation along the $j$th axis,\footnote{In particular, $\omega_{j,0}$ would correspond to the unscreened plasma frequency $\omega_{p,0}=\sqrt{ne^2/m\e_0}$ in an isotropic, free-electron model.  Due to the screening associated with $\e_{j,\infty}$, the physical plasma frequency along the $j$th axis in \eqref{dielectric component Drude definition} is suppressed as $\omega_{j,0}/\sqrt{\e_{j,\infty}}$.} and $\Gamma$ is the phenomenological scattering rate.  The choice of an isotropic scattering rate is for convenience; most of the subsequent study will focus on well-defined bulk excitations ($\omega\gg\Gamma$), where anisotropic scattering is presumed to not be the dominant effect.

\note{Todo - consider anisotropic scattering?}


\subsection{Bulk response: plasmon excitation}

The dielectric tensor $\e$ \eqref{dielectric tensor definition} governs the electrostatic response of the system through Gauss's law.  In a translation-invariant geometry,
\begin{equation}
    \label{translation invariant Gauss's law}
    \e_L(\hat q,\omega) \phi(q,\omega) = \frac{\rho_{ext}(q,\omega)}{q^2 \e_0}\,\,\,,
\end{equation}
where $q$ is the wavevector, $\hat q$ is the {\it unit} wavevector, $\omega$ is the frequency, $\phi$ is the (total) electric potential, $\rho_{ext}$ is the free charge density external to the system under study, $\e_0$ is the permitivitty of free space due to working in S.I. units, and we've defined the {\it longitudinal} dielectric function,
\begin{equation}
    \label{longitudinal dielectric function}
    \e_L(\hat q,\omega) := \hat q\cdot \e(\omega)\cdot \hat q\,\,\,.
\end{equation}
From Gauss's law \eqref{translation invariant Gauss's law}, the system contains a resonant, self-sustaining electrostatic excitation whenever $\e_L$ of \eqref{longitudinal dielectric function} vanishes.  As in the typical free-electron Drude model, this mode corresponds to the plasmon in the anisotropic Drude model of \eqref{dielectric component Drude definition}.

The nature of bulk electrostatic excitations associated with a model dielectric function can be studied through the {\it loss function},
\begin{equation}
    \label{bulk loss function}
    -\im \e^{-1}_L(\hat q,\omega) = -\im\frac{1}{\e_L(\hat q,\omega)}
    \,\,\,.
\end{equation}
In the regime of well-defined plasmon excitation ($\Gamma \ll \omega$ for $\omega$ near the plasma frequency), the plasmon dispersion roughly follows from zeros in the real part of the longitudinal dielectric function \eqref{longitudinal dielectric function}.  As a first approximation,
\begin{equation}
    \label{real part of longitudinal dielectric function without damping}
    \re\e_L(\hat q, \omega) \approx \e_L(\hat q, \omega)\Big|_{\Gamma = 0}
    =
    \sum_{j=x,y,z} \hat q_j^2\lp\e_{j,\infty} - \frac{\omega_{j,0}^2}{\omega^2}\rp
    \,\,\,,
\end{equation}
which determines the plasmon orientation dependence $\omega_p(\hat q)$ through the positive root,
\begin{equation}
    \label{approximate plasmon orientation}
    \omega_p(\hat q) \approx 
    \sqrt{\frac{\sum_{j=x,y,z} \hat q_j^2 \omega_{j,0}^2}{\hat q \cdot \e_{\infty} \cdot \hat q}}
    \,\,\,.
\end{equation}
We should expect from \eqref{approximate plasmon orientation} that plasmon excitation smoothly varies from its value along one axis to another as the orientation is rotated between the two.

\note{[Add Gumhalter/other references]}: This likely needs lots of references about this anisotropic dielectric function and the simple plasmon dispersion that results.

In Figure~\ref{figure: anisotropic plasmon spectrum}, the loss function \eqref{bulk loss function} is plotted for the anisotropic Drude model as a function of frequency $\omega$ and orientation $\theta$.  The wavevector orientation $\hat q$ was chosen to lie within the $xy$-plane, where $\theta$ is the azimuthal angle measured counterclockwise from the $x$-axis.  

\begin{figure}
    \centering
    % \includegraphics[width=0.5\columnwidth]{figures/superlattice-geometry.pdf}
    \caption{
        \note{Todo}: create plot and caption.
    }
    \label{figure: anisotropic plasmon spectrum}
\end{figure}
