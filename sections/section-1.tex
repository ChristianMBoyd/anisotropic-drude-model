\section{The anisotropic Drude model}

This paper examies an anisotropic and homogeneous dielectric tensor,
\begin{equation}
    \label{dielectric tensor definition}
    \e(\omega) = \bpm
    \ex(\omega) & 0 & 0
    \\ 0 & \ey(\omega) & 0
    \\ 0 & 0 & \ez(\omega)
    \epm
    \,\,\,,
\end{equation}
where the diagonal entries of $\e$ \eqref{dielectric tensor definition} resemble the Drude model,
\begin{equation}
    \label{dielectric component Drude definition}
    \e_j(\omega) = \e_{j,\infty}\lp 1-\frac{\omega_{p,j}^2}{\omega\lp\omega+i\Gamma\rp}\rp\,\,\,.
\end{equation}
In \eqref{dielectric component Drude definition}, $j\in(x,y,z)$, $\e_{j,\infty}$ phenomenologically models inter-band or background dielectric contributions oriented along the $j$th axis, $\omega_{p,j}$ is the frequency of plasma oscillation along the $j$th axis,\footnote{In particular, $\omega_{p,j}$ denotes the physical plasma frequency that would be observed for charge density oscillations along the $j$th axis, not the unscreened value $\omega_{p,0}=\sqrt{ne^2/m\e_0}$.} and $\Gamma$ is the phenomenological scattering rate.  The choice of an isotropic scattering rate is for convenience; most of the subsequent study will focus on well-defined bulk excitations ($\omega\gg\Gamma$), where anisotropic scattering is presumed to not be the dominant effect.

\note{Todo - consider anisotropic scattering?}





\subsection{Plasmon excitation}

The bulk electrostatic excitations of the anisotropic dielectric tensor $\e$ of \eqref{dielectric tensor definition} can be studied through the {\it loss function},
\begin{equation}
    -\im \e^{-1}(\hat q, \omega) = \hat q \cdot \e^{-1}(\omega) \cdot \hat q
    \,\,\,,
\end{equation}
where $\hat q$ is a unit wavevector due to the lack of wavevector dependence in $\e$.  Inverting $\e$,
\begin{equation}
    \label{bulk loss function}
    -\im\e^{-1}(\hat q, \omega) = \sum_{j=x,y,z} \lp\frac{q_j^2}{q^2}\rp \lb -\im\e_{jj}^{-1}(\omega)\rb
    =
    \sum_{j=x,y,z}\lp\frac{q_j^2}{q^2}\rp\frac{\im\e_{jj}(\omega)}{|\e_{jj}(\omega)|^2}
    \,\,\,.
\end{equation}
In the regime of well-defined plasmon excitation, $\Gamma \ll \omega_j$ for all $j=x,y,z$, the plasmon dispersion roughly follows from zeros in the real part of the {\it longitudinal} dielectric function,
\begin{equation}
    \e_L(\hat q,\omega) :=\hat q\cdot \e\cdot\hat q\,\,\,,
\end{equation}
where $\hat q$ is a unit wavevector enforcing excitations along the longitudinal direction.  As a first approximation,
\begin{equation}
    \label{longitudinal dielectric function without damping}
    \re\e_L(\hat q, \omega) \approx \e_L(\hat q, \omega)\Big|_{\Gamma = 0}
    =
    \sum_{j=x,y,z}q_j^2\e_{j,\infty}\lp1 - \frac{\omega_j^2}{\omega^2}\rp
    \,\,\,,
\end{equation}
which roughly provides the plasmon dispersion $\omega_p(\hat q)$ in terms of a simple formula,
\begin{equation}
    \omega_p(\hat q) \approx 
    \frac{\sum_{j=x,y,z}q_j^2 \omega_j^2}{\sum_{j=x,y,z}q_j^2 \e_{j,\infty}}
    \,\,\,.
\end{equation}
\note{Add Gumhalter/other references}: This likely needs lots of references about this kind of dielectric function.

In Figure~\ref{figure: anisotropic plasmon spectrum}, the loss function \eqref{bulk loss function} is plotted for the anisotropic Drude model as a function of frequency $\omega$ and orientation $\theta$.  The wavevector orientation $\hat q$ was chosen to lie within the $xy$-plane, where $\theta$ is the azimuthal angle measured counterclockwise from the $x$-axis.  

\begin{figure}
    \centering
    % \includegraphics[width=0.5\columnwidth]{figures/superlattice-geometry.pdf}
    \caption{
        \note{Todo}: create figure/caption.
    }
    \label{figure: anisotropic plasmon spectrum}
\end{figure}
